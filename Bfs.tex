//Breadth-First Search (BFS) algorithm

#include<stdio.h>
#include<stdlib.h>
 struct queue
 {
     int size;
      int f;
       int r;
        int* arr;
      };
  int isEmpty(struct queue *q)
  {
      if(q->r==q->f)
        {
             return 1;
      }
   return 0;
    }
   int isFull(struct queue *q)
   {
        if(q->r==q->size-1)
            {
                 return 1;
         }
 return 0;
  }
   void enqueue(struct queue *q, int val)
   {
        if(isFull(q))
   {
       printf("This Queue is full\n");
        }
        else
            {
                 q->r++; q->arr[q->r] = val;
  printf("Enqued element: %d\n", val);
   }
    }
    int dequeue(struct queue *q)
    {
        int a = -1;
         if(isEmpty(q))
            {
printf("This Queue is empty\n");
 }
  else{ q->f++; a = q->arr[q->f];
  }
 return a;
   } int main(){
       struct queue q;
        q.size = 400;
        q.f = q.r = 0;
        q.arr = (int) malloc(q.size*sizeof(int));
  int node;
   int i = 3;
    int visited[7] = {0,0,0,0,0,0,0};
     int a [7][7] = {
         {0,1,1,1,0,0,0},
      {1,0,1,0,0,0,0},
      {1,1,0,1,1,0,0},
       {1,0,1,0,1,0,0},
        {0,0,1,1,0,1,1},
       {0,0,0,0,1,0,0},
        {0,0,0,0,1,0,0}
       }; printf("%d", i);
        visited[i] = 1;
        enqueue(&q, i);
        while (!isEmpty(&q))
            {
                int node = dequeue(&q);
        for (int j = 0; j < 7; j++)
            {
                 if(a[node][j] ==1 && visited[j] == 0)
                 {
                      printf("%d", j);
        visited[j] = 1;
        enqueue(&q, j);
        }
         }
         }
         return 0;
          }
